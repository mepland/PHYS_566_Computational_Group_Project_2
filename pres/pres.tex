\documentclass[mathserif,18pt,xcolor=table]{beamer}
\usepackage{amsmath}
\usepackage{amssymb}
\usepackage{bbm}
\usepackage{ulem}
\usepackage{feynmp-auto}
%\usepackage{slashed}
\usepackage[absolute,overlay]{textpos}
\usepackage{graphicx}
\usepackage{listings}
\usepackage{epsfig}
\usepackage{hyperref}
\usepackage{tikz}
\usetikzlibrary{calc}
\usepackage{enumerate}
%\usepackage{fixltx2e} % buggy
\usepackage[compatibility=false]{caption}
\usepackage{subcaption} % doesn't work with subfigure
%\usepackage{pdfpages}
\usepackage{setspace}
\usepackage{verbatim}
\usepackage{physics}
%\usepackage{siunitx}

\DeclareRobustCommand{\orderof}{\ensuremath{\mathcal{O}}}

\definecolor{dukeblue}{RGB}{0,0,156}
\definecolor{dukedarkblue}{RGB}{0,26,87}
\definecolor{dukeblack}{RGB}{79,79,79}
\definecolor{dukegray}{RGB}{79,79,79}
\definecolor{dukesecbrown}{RGB}{217,200,158}
\definecolor{dukesecblue}{RGB}{127,169,174}
\mode<presentation> {
  \usetheme{Boadilla}  
  \setbeamercovered{invisible}
  \setbeamertemplate{navigation symbols}{}  
  \setbeamertemplate{frametitle}[default][center]
  \setbeamertemplate{bibliography item}{\insertbiblabel}
  \setbeamerfont{frametitle}{series=\bfseries,parent=structure}
  \setbeamerfont{subtitle}{size=\scriptsize,series=\bfseries,parent=structure}
  \setbeamerfont{author}{size=\scriptsize,parent=structure}
  \setbeamerfont{institute}{size=\small,series=\bfseries,parent=structure}
  \setbeamerfont{date}{size=\scriptsize,parent=structure}
  \setbeamerfont{footline}{size=\tiny,parent=structure}
  \setbeamercolor{normal text}{bg=white,fg=dukeblack}
  \setbeamercolor{structure}{fg=dukeblue}
  \setbeamercolor{alerted text}{fg=red!85!black}
  \setbeamercolor{item projected}{use=item,fg=black,bg=item.fg!35}
  \setbeamercolor*{palette primary}{use=structure,fg=white, bg=dukeblue}
  \setbeamercolor*{palette secondary}{use=structure,bg=dukedarkblue,fg=white}
  \setbeamercolor*{framesubtitle}{fg=dukegray}
  \setbeamercolor*{block title}{parent=structure,fg=black,bg=dukeblue}
  \setbeamercolor*{block body}{fg=black,bg=dukeblack!10}
  \setbeamercolor*{block title alerted}{parent=alerted text,bg=black!15}
  \setbeamercolor*{block title example}{parent=example text,bg=black!15}
}

\makeatletter
\setbeamertemplate{footline}{
  \leavevmode
  \hbox{%
    \begin{beamercolorbox}[wd=.333333\paperwidth,ht=2.25ex,dp=1ex,center]{author in head/foot}%
      \usebeamerfont{author in head/foot}\insertshortauthor%
    \end{beamercolorbox}%
    \begin{beamercolorbox}[wd=.333333\paperwidth,ht=2.25ex,dp=1ex,center]{title in head/foot}%
      \usebeamerfont{title in head/foot}\insertshorttitle%
    \end{beamercolorbox}%
    \begin{beamercolorbox}[wd=.333333\paperwidth,ht=2.25ex,dp=1ex,right]{date in head/foot}%
      \usebeamerfont{date in head/foot}\insertshortdate{}\hspace*{2em}%
      \insertframenumber{} / \inserttotalframenumber\hspace*{2ex}%
%      \insertframenumber{} / 11\hspace*{2ex}% TODO hard code to page number before backup!!
    \end{beamercolorbox}}%
  \vskip0pt%
}
\makeatother


\AtBeginSection{\frame{\sectionpage}}


\defbeamertemplate{section page}{mine}[1][]{%
  \begin{centering}
    {\usebeamerfont{section name}\usebeamercolor[fg]{section name}#1}
    \vskip1em\par
    \begin{beamercolorbox}[sep=12pt,center]{part title}
      \usebeamerfont{section title}\insertsection\par
    \end{beamercolorbox}
  \end{centering}
}


\usepackage[protrusion=true,expansion=true]{microtype}
\usepackage{amsmath}
\renewcommand*{\thefootnote}{\fnsymbol{footnote}}
\title[Group Project 1]{Group Project 1\newline Random Walk, Diffusion and Cluster Growth}
\author[Xu, Epland, Li, Cohen]{{\small Yuanyuan Xu, Matthew Epland, Xiaqing Li, Wesley Cohen}}
\institute{Duke University}
%\date{\today}
\date{March 25, 2016}
\hypersetup{
    breaklinks,
    baseurl       = http://,
    pdfborder     = 0 0 0,
    pdfpagemode   = UseNone,% do not show thumbnails or bookmarks on opening
    pdfstartpage  = 1,
    bookmarksopen = true,
    bookmarksdepth= 2,% to show sections and subsections
    pdfauthor     = {\@author},
    pdftitle      = {\@title},
    pdfsubject    = {},
    pdfkeywords   = {}}

\titlegraphic{\includegraphics[height=2cm]{logos/duke_logo.pdf}}

% Point to nice top level directory
\graphicspath{{../output/plots_for_paper/}}

\begin{document}

\beamertemplateballitem
\frame{\titlepage}
\addtobeamertemplate{frametitle}{}{}


%%%%%%%%%%%%%%%%%%%%%%%%%%%%%%%%%%%%%%%%%%%%%%%%%%%%%%%%%%%%%%
% Problem 1
%%%%%%%%%%%%%%%%%%%%%%%%%%%%%%%%%%%%%%%%%%%%%%%%%%%%%%%%%%%%%%
\begin{frame}
	\frametitle{Problem 1 Overview}
	\begin{itemize}
		\item In Part A, we used the random number generator to determine the direction of the walker's motion. We used two for loops one for the number of walks and one for the number of steps in a single random walk. 
		\item In Part B, we used the polyfit function to fit a linear function to the mean square distance data.
	\end{itemize}
\end{frame}

\begin{frame}
	\frametitle{Part A (1/3)}
	{
	\tt
	\qquad n = 100     \\                                 
	\qquad m = 10000.0   \\                             
	~\\
	\qquad x2ave=np.asarray([0.0]*n) \\                
	\qquad y2ave=np.copy(x2ave)      \\                 
	\qquad xAve=np.copy(x2ave)       \\                  
	\qquad dist2=np.copy(x2ave)                        
	}
\end{frame}

\begin{frame}
	\frametitle{Part A (2/3)}
	{
	\tt
	\qquad for j in range(int(m)):   	\\
    \qquad \qquad x = 0               	\\        
    \qquad \qquad y = 0               	\\    
    \qquad \qquad for i in range(n):    	 	\\
    \qquad \qquad \qquad r = random.random() 	\\
    \qquad \qquad \qquad if r <= 0.25:       	\\
    \qquad \qquad \qquad\qquad x += 1     		\\
    \qquad \qquad \qquad elif r <= 0.5:		 	\\
    \qquad \qquad \qquad\qquad x -= 1			\\
    \qquad \qquad \qquad elif r <= 0.75:		\\
    \qquad \qquad \qquad\qquad y += 1	 		\\
    \qquad \qquad \qquad else:				 	\\
    \qquad \qquad \qquad \qquad y -= 1			\\
    \qquad \qquad \qquad xAve[i] += x   		\\                         
    \qquad \qquad \qquad x2ave[i] += x ** 2    \\
    \qquad \qquad \qquad dist2[i] += x ** 2 + y ** 2
	}
\end{frame}

\begin{frame}
	\frametitle{Part A (3/3)}
	{
	\tt
	\qquad xAve /= m  				\\
    \qquad x2ave /= m               	\\        
    \qquad dist2 /= m               	\\
	~\\    
    \qquad distNew = dist2[3:100]   \\                     
	\qquad xAveNew = xAve[3:100]  	\\                      
	\qquad x2aveNew = x2ave[3:100]  \\                
	}	
\end{frame}

\begin{frame}
	\frametitle{Part B}
	\tt
	{
	\qquad steps = np.arange(4, n + 1, 1)    \\               
	~\\
	\qquad coefficients = np.polyfit(steps, distNew, 1)   \\
	\qquad slope = coefficients[0]           \\          
	\qquad diffCoeff = slope/4                          	\\
  	~\\
	\qquad eq = np.poly1d(coefficients) \\          
	\qquad eqSteps = eq(steps)    \\                
	}
\end{frame}

\begin{frame}
	\frametitle{$\expval{x_n}$ Plot}
	\begin{figure}
  		\centering
  		\includegraphics[width=0.9\textwidth]{../output/plots_for_paper/problem_1/xn_Plot.pdf}
	\end{figure}
\end{frame}

\begin{frame}
	\frametitle{$\expval{x^{2}_n}$ Plot}
	\begin{figure}
  		\centering
  		\includegraphics[width=0.9\textwidth]{../output/plots_for_paper/problem_1/xn2_Plot.pdf}
	\end{figure}
\end{frame}

\begin{frame}
	\frametitle{$\expval{r^{2}}$ Plot}
	\begin{figure}
  		\centering
  		\includegraphics[width=0.9\textwidth]{../output/plots_for_paper/problem_1/Diffusion_Plot.pdf}
	\end{figure}
	\begin{textblock*}{0.9\textwidth}(8.5cm,6.5cm)
		{\color{red} slope $\approx 1$ \\ $D \approx 0.25$}	
	\end{textblock*}
\end{frame}

%%%%%%%%%%%%%%%%%%%%%%%%%%%%%%%%%%%%%%%%%%%%%%%%%%%%%%%%%%%%%%
% Problem 2
%%%%%%%%%%%%%%%%%%%%%%%%%%%%%%%%%%%%%%%%%%%%%%%%%%%%%%%%%%%%%%

\begin{frame}
	\frametitle{Random Walk and Diffusion}
	Suppose $P(i, j, k, n)$ is the probability to find the particle at $(x, y, z)$ at time $t$, then we can get
\begin{equation}
	\scriptsize
	\begin{split}
		P(i, j, k) = \frac{1}{6}[&P(i, j, k, n-1) + P(i-1, j, k, n-1) + P(i, j+1, k, n-1) +\\ &P(i, j-1, k, n-1) + P(i, j, k+1, n-1) + P(i, j, k-1, n-1)].
	\end{split}
\end{equation}
Rearrange this equation and take the continuum limit,
\begin{equation}
	\frac{\partial P(x, y, z, t)}{\partial t} = D\nabla^{2}P(x,y,z,t),
\end{equation}
where $D = (1/6)(\Delta x)^{2} / \Delta t$. 
\end{frame}

\begin{frame}
	\frametitle{Diffusion Equation}
	Suppose the density $\rho$ obeys the diffusion equation,
	\begin{equation}
		\frac{\partial \rho}{\partial t} = D\nabla^{2}\rho.
	\end{equation}
	In one dimension, we have
	\begin{equation}
		\frac{\partial\rho}{\partial t} = D\frac{\partial^{2}\rho}{\partial x^{2}}.
	\end{equation}
	One special solution is
	\begin{equation}
		\rho(x, t) = \frac{1}{\sigma}e^{-\frac{x^{2}}{2\sigma^{2}}},
	\end{equation} 
	where $\sigma$ is time-dependent, with $\sigma = \sqrt{2Dt}$. \\
	And the finite-difference version of this equation is
	\begin{equation}
	\frac{\rho(i, n+1) - \rho(i, n)}{\Delta t} = \frac{\rho(i+1, n) - 2\rho(i, n) + \rho(i-1, n)}{(\Delta x)^{2}}
\end{equation}
\end{frame}

\begin{frame}
	\frametitle{$\expval{x^{2}(t)}$ of 1D Normal Distribution}
	The formula for computing the expectation value is given by
	\begin{align}
	\expval{x^{2}(t)} &= \int_{-\infty}^{+\infty} \rho(x, t) x^{2} dx, \\
	\expval{x^{2}(t)} &= \frac{2}{\sqrt{2\pi\sigma^{2}(t)}} \int_{0}^{+\infty} x^{2} \exp(-\frac{x^{2}}{2\sigma^{2}(t)}).
\end{align}
Now we use the Gaussian integral that
\begin{equation}
	\int_{0}^{+\infty} x^{2n} \exp(-x^{2}/a^{2}) dx = \sqrt{\pi} \frac{(2n)!}{n!}\Big( \frac{a}{2} \Big)^{2n+1}.
\end{equation}
Let $n = 1$ and $a = \sqrt{2}\sigma(t)$, we can get
\begin{equation}
	\begin{split}
		\expval{x^{2}(t)} &= \frac{2}{\sqrt{2\pi\sigma^{2}(t)}} \cdot \sqrt{\pi}\cdot \frac{2!}{1!} \cdot \Big( \frac{\sqrt{2}\sigma(t)}{2} \Big)^{3} \\
		&= \sigma^{2}(t).
	\end{split} 
\end{equation}
\end{frame}

\begin{frame}
	\frametitle{Numerical Methods}
	\begin{itemize}
		\item Set spatial and time step size
		\begin{itemize}
		\item
		$\Delta x = 0.05, \Delta t = 10^{-4}$ which satisfy $\Delta t \leq (\Delta x)^{2}/(2D)$
		\end{itemize}
		\item Set boundary conditions
	    \begin{itemize}
	    \item
	    	$\rho(-5) = \rho(5) = 0$
	    \end{itemize}
	    \item Set initial conditions
	    \begin{itemize}
	    	\item $\rho(x) = \begin{cases}
	    	1/(3\Delta x) & x = -\Delta x, 0, \Delta x \\
	    	0 & \mathrm{others}\\
	    	\end{cases}$
	    \end{itemize}
	    \item Loop over time steps and spatial steps
	    \begin{itemize}
	    	\item {
	    	\tt	
	    	for i in range(t\_max): \\
	    	\qquad for j in range(size):\\
	    	\qquad \qquad $\rho$[j] += D*dt*($\rho$[j+1] - 2*$\rho$[j] + $\rho$[j-1])/dx$^{2}$
	    	}
	    \end{itemize}
	    \item Perform a fit for 5 different time snapshots
	    \item \href{run:movie.avi}{Evolution of a Gaussian}
	\end{itemize}
\end{frame}

\begin{frame}
	\frametitle{Snapshots}
	\begin{figure}
  	\centering
  	\includegraphics[width=0.85\textwidth]{../output/plots_for_paper/problem_2/part_b.pdf}
\end{figure}
\end{frame}

\begin{frame}
	\frametitle{Verification}
	\begin{center}
	\begin{tabular}{ | p{1.5cm} | p{1.5cm} | p{2.5cm} | p{1.5cm} | p{2.5cm} |}
		\hline
		Steps & Time & $\sigma(t) = \sqrt{2Dt}$ & $\sigma_{\mathrm{fit}}(t)$ & Percent error\\
		\hline
			300  & 0.03 & 0.346 & 0.363 & 4.716\%\\
			\hline
			600  & 0.60 & 0.490 & 0.512 & 4.486\%\\ 
			\hline
			1200 & 1.20 & 0.693 & 0.723 & 4.371\%\\
			\hline
			2400 & 2.40 & 0.980 & 0.102 & 4.314\%\\
			\hline
			4800 & 4.80 & 0.139 & 0.144 & 4.279\%\\
			\hline
	\end{tabular}
\end{center}
\begin{itemize}
	\item $\sigma(t) = \sqrt{2Dt}$
	\item $\expval{r^{2}} \propto t$
\end{itemize}
\end{frame}


%%%%%%%%%%%%%%%%%%%%%%%%%%%%%%%%%%%%%%%%%%%%%%%%%%%%%
\begin{frame}
  \frametitle{Problem 3: Introduction}
  \begin{itemize}
    \item Diffusion-limited aggregation (DLA) model
    \begin{itemize}
    	\item Seed located at the origin
	\item Random walker released at random point on a circle of $R=100$
	\item Halting conditions -- when the random walker
	\begin{itemize}
		\item contacted an existing cluster point $\Rightarrow$ form a new cluster point
		\item traveled too far away from the cluster ($\mathrm{min}\left|\mathbf{r}_{\mathrm{Cluster\,Point}} - \mathbf{r}_{\mathrm{Walker}}\right|  > d_{\mathrm{Kill}} = 120$)
		\item walked for too many steps ($t > t_{\mathrm{Kill}} = 10^{4}$)
	\end{itemize}
    \end{itemize}
    \item Fractal dimensionality $d_f$
    \begin{itemize}
    	\item $m(r)\sim r^{d_f}$
    	\item $\log m\sim {d_f}\log r$
    \end{itemize}
  \end{itemize}
\end{frame}

\begin{frame}
  \frametitle{DLA Cluster: RNG Seed $= 6$}
\begin{figure}
  \centering
  \includegraphics[width=0.9\textwidth]{problem_3/large_cluster_seed_num_6.pdf}
\end{figure}
\end{frame}

\begin{frame}
  \frametitle{DLA Cluster Mass: RNG Seed $= 6$}
\begin{figure}
  \centering
  \includegraphics[width=0.9\textwidth]{problem_3/large_cluster_mass_seed_num_6.pdf}
\end{figure}
\end{frame}

\begin{frame}
  \frametitle{DLA Cluster: RNG Seed $= 11$}
\begin{figure}
  \centering
  \includegraphics[width=0.9\textwidth]{problem_3/large_cluster_seed_num_11.pdf}
\end{figure}
\end{frame}

\begin{frame}
  \frametitle{DLA Cluster Mass: RNG Seed $= 11$}
\begin{figure}
  \centering
  \includegraphics[width=0.9\textwidth]{problem_3/large_cluster_mass_seed_num_11.pdf}
\end{figure}
\end{frame}

\begin{frame}
  \frametitle{DLA Cluster: RNG Seed $= 14$}
\begin{figure}
  \centering
  \includegraphics[width=0.9\textwidth]{problem_3/large_cluster_seed_num_14.pdf}
\end{figure}
\end{frame}

\begin{frame}
  \frametitle{DLA Cluster Mass: RNG Seed $= 14$}
\begin{figure}
  \centering
  \includegraphics[width=0.9\textwidth]{problem_3/large_cluster_mass_seed_num_14.pdf}
\end{figure}
\end{frame}

\begin{frame}
  \frametitle{Fractal Dimensionality}

  \begin{itemize}
    \item Running over 10 different RNG seeds we find:
    \begin{itemize}
      \item $d_f$ of 10 different DLA clusters: $[1.608, \,1.805, \,1.700, \,1.609, \,1.994, \,1.641, \,1.479, \,1.536, \,1.642, \,1.595]$
      \item $\left<d_{f}\right> = 1.66$
      \item $\sigma_{d_f}= 0.14$ 
    \end{itemize}
  \end{itemize}

% You can position text/figures/anyting at an arbitrary position with textblock*
%\begin{textblock*}{0.9\textwidth}(1.2cm,5.5cm) % {block width} (coords) note coords are from upper left corner
%{\tiny All $d_{f\,\mathrm{fit}} = 1.3347,\,1.3463,\,1.4241,\,1.3854,\,1.0841,\,1.3420,\,1.3673,\,1.2641,\,1.2410,\,1.1970$}

\end{frame}


\end{document}
%%%%%%%%%%%%%%%%%%%%%%%%%%%%%%%%%%%%%%%%%%%%%%%%%%%%%%%%%%%%%%%%%%%%

% Backup
\begin{frame}
    \begin{center}
    \usebeamerfont{frametitle}Backup
    \end{center}
\end{frame}

% bib
\begin{frame}%[allowframebreaks]
        \frametitle{References}
	\bibliographystyle{../bib_files/atlasBibStyleWoTitle}
{\scriptsize
        \bibliography{../bib_files/my_bib.bib}
}
\end{frame}


