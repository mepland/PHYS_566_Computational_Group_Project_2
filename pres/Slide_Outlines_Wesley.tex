\documentclass[12pt]{article}
\usepackage{amsmath}
\usepackage{amssymb}
\usepackage{graphicx}
\begin{document}

\section{Title Slide}

\section{Overview of Part A}
\noindent Lattice dimensions of N=5, 10, 15, 20, 30, 50, 80 were used. \newline

\noindent The results were averaged over 50 simulations. \newline

\noindent The critical probability $p_c$ was determined. \newline

\noindent The infinite size limit $p_c(0)$ was determined. 

\section{Equations for Part A}

\noindent Critical Probability:

\begin{equation}
p_c=\frac{\text{number of occupied sites}}{\text{total number of sites}}
\end{equation}

\section{Program Structure for Part A}

\section{pc(N-1) Plot}

\section{Conclusions for Part A}

\section{Overview of Part B}
\noindent Lattice dimensions of N=5, 10, 15, 20, 30, 50, 80 were used. \newline

\noindent The lattice size is fixed at N=100. \newline

\noindent The fraction of sites is calculated. \newline

\noindent 50 simulations are used. \newline

\noindent The results are fit to a Power-Law Ansatz.

\section{Equations for Part B}

\noindent Fraction of Sites:

\begin{equation}
F(p>p_c)=\frac{\text{number of sites in spanning cluster}}{\text{number of occupied sites}}
\end{equation}

\noindent Power-Law Ansatz:

\begin{equation}
F=F_0(p-p_c)^\beta
\end{equation}

\section{Program Structure for Part B}

\section{Log-Log Plot}

\section{Conclusions for Part B}

\section{Questions}


\end{document}




